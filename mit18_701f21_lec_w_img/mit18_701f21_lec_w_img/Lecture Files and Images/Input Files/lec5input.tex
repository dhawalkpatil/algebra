%MIT OpenCourseWare: https://ocw.mit.edu
%RES.18-011 Algebra I Student Notes, Fall 2021
%License: Creative Commons BY-NC-SA 
%For information about citing these materials or our Terms of Use, visit: https://ocw.mit.edu/terms.

\section{The Correspondence Theorem}
\subsection{Review}

In the last lecture, we learned about cosets and some of their properties. 

\begin{definition}
For a group $G$ and a subgroup $H \leq G,$ we define the \textbf{left coset} of $a$ to be 
\[
aH \coloneqq \{ah : h \in H\} \subseteq G.
\]
\end{definition}

The left cosets \emph{partition}\footnote{A partition of a set $S$ is a subdivision of the entire set into disjoint subsets.} $G$ into equally sized sets. This provides a useful corollary about the structure of cosets within a group:

\begin{corollary}[Counting Formula.]
Let $[G:H]$ be the number of left cosets of $H$, which is called the \textbf{index} of $H$ in $G.$ Then $|G| = |H|[G:H]$.
\end{corollary}

\subsection{Lagrange's Theorem}
Using cosets provides some additional information about groups.
\begin{qq}
What are the possibilities for the structure of a group with order $n$?
\end{qq}

From the Counting Formula, we immediately obtain Lagrange's Theorem as a corollary:
\begin{theorem} [Lagrange's Theorem.]
For $H$ a subgroup of $G,$ $|H|$ is a divisor of $|G|.$
\end{theorem}

Several important corollaries follow as a result.
\begin{corollary}
The order of $x \in G$ is $|\langle x \rangle|.$ Since the order of any subgroup divides the order of $|G|,$ $\ord(x)$ also divides $|G|$.
\end{corollary}

\begin{corollary}
Any group $|G|$ with prime order $p$ is a cyclic group. 
\end{corollary}
\begin{proof}
Take an element $e \neq x \in G.$ Since the order of $x \in G$ divides $p,$ and $p$ is prime, $\ord(x) = p.$ Then each $x^i$ is distinct for $0 \leq i \leq p-1,$ and since there are only $p$ elements in $G,$ the entire group $G$ is $\langle x \rangle,$ the cyclic group generated by $x.$
\end{proof}

Our result shows that any group of prime order is a cyclic group. In particular, the integers modulo $p$, $\ZZ_p$, form a cyclic group of prime order; that is, any group of prime order $p$ is isomorphic to $\ZZ_p.$ 

\subsection{Results of the Counting Formula}
Using Lagrange's Theorem narrows down the possibilities for subgroups. 
\begin{example}[Groups of order 4.]
What are the possibilities (up to isomorphism) for $G$ if $|G| = 4$?

First, $e$ must be an element of $G$. Next, consider the other three elements of $G$. Each of these must have either order $2$ or order $4,$ since those are the divisors of $|G| = 4.$ Then there are two possibilities. 

%since $G$ has four elements, there exists some element $e \neq x \in G.$ Because $\ord(x)$ must divide $|G|$, $\ord(x) = 2$ or $\ord(x) = 4.$ (only the identity has order 1). There are two cases. 
\begin{itemize}
    \item \textbf{Case 1.} There exists an element $x \in G$ such that $\ord(x) = 4$. Then we know that $e \neq x \neq x^2 \neq x^3,$ and since $|G| = 4,$ these are all the elements of $G$. (The power $x^4$ is $e$ again.) So $G$ is generated by $x,$ and it is the cyclic group $\langle x \rangle$ of size 4, and must be isomorphic to $\ZZ_4$\footnote{We write $\ZZ_n$ to denote the group of integers modulo $n$.}.
    
    \item \textbf{Case 2.} All elements of $G$ have order 2. Then, we can take $x \in G$ and $y \neq x \in G.$ They have order 2, so $x^2 = e$, which implies that $x = x^{-1}$ and similarly $y = y^{-1}.$ Also, the element $xy$ also has order 2, and so $xyx^{-1}y^{-1} = (xy)(xy) = e,$ and so $x$ and $y$ commute. Because $x$ and $y$ were chosen arbitrarily, any two elements of the group commute, and so it is abelian. 

    This group $G$ is isomorphic to the matrix group
    \[
    \begin{pmatrix}
    \pm 1 & 0 \\
    0 & \pm 1
    \end{pmatrix} 
    \leq GL_2(\RR).
    \]
    
    The non-identity elements each have order 2 and commute with each other. This group is called the Klein-four group, and is denoted $K_4.$
\end{itemize}

Up to isomorphism, any order 4 group is either $\ZZ_2$ or $K_4.$ Note that both of these groups are abelian\footnote{commutative}; the smallest non-abelian group has order 6.

\end{example}

\begin{exercise}
What are the possible groups of order 6?
\end{exercise}

The Counting Formula also provides another important corollary.
\begin{corollary}
The size of the group is 
\[
|G| = \lvert\ker(f)\rvert\cdot\lvert\im(f)\rvert.\footnote{In linear algebra, the analogous result is the rank-nullity theorem.}
\]
\end{corollary}
\begin{proof}
Let $f: G \rightarrow G'$ be a homomorphism, and $\ker(f) \leq G$ be the kernel. For each $y \in G',$ the preimage of $y$ is 
\[
f^{-1}(y) \coloneqq \{x \in G: f(x) = y\},
\]
which is $\varnothing$ if $y \notin \im(f)$, and a coset of $\ker(f)$ otherwise.\footnote{Pick $x \in f^{-1}(y).$ Then we claim that $f^{-1}(y) = x\ker(f).$ Take any $x' \in f^{-1}(y).$ We have $y = f(x) = f(x') = f(xx'^{-1})f(x'),$ so $f(xx'^{-1}) = e$ and $xx'^{-1} \in \ker(f).$ Thus $x' \in x\ker(f).$}
\end{proof}

Then, the number of left cosets of $\ker(f)$ is precisely the number of elements in the image of $f,$ since each of those elements corresponds to a coset of the kernel. So $[G: \ker(f)] = \lvert\im(f)\rvert,$ and applying the counting formula with $\ker(f)$ as our subgroup $H$ gives us 
\[
|G| = \lvert\ker(f)\rvert\cdot\lvert\im(f)\rvert,
\]
which is the desired result. 

\subsection{Normal Subgroups}

In this section, we learn about normal subgroups.
\begin{qq}
The choice of left cosets seems arbitrary — what are the ramifications if \emph{right cosets} are used instead?
\end{qq}
\begin{definition}
The \textbf{right coset} of $a$ is 
\[
Ha = \{ha : h \in H\}.
\]
\end{definition}

In fact, all the same results follow if right cosets are used instead of left cosets. First, let's see an example of right cosets:
 
\begin{example}
Let $H$ be the subgroup generated by $y \in S_3.$ Then the left cosets are 
\[
\{e, y\}, \{x, xy\}, \{x^2, x^2y\},
\]
and the right cosets are 
\[
\{e, y\}, \{x, x^2y\}, \{x^2, xy\}.
\]

So in fact, right cosets give a different partition of $S_3,$ but the number and size of the cosets are the same.\footnote{We can think of cosets as "carving up" the group. Using right cosets instead of left cosets is just carving it up in a different way.}
\end{example}

In particular, there is a bijection between the set of left cosets and the set of right cosets. It maps

\[
C \mapsto C^{-1} = \{x^{-1} : x \in C\}.
\]
It is a bijection because $(ah)^{-1} = h^{-1}a^{-1},$ and so $aH = Ha^{-1}$. So the index $[G:H]$ is equal to both the number of right cosets and the number of left cosets. 

\begin{qq}
For which subsets $H \subseteq G$ do left and right cosets give the \textbf{same} partition of $G$? In other words, for which $H$ is every left coset also a right coset?\footnote{If some left coset $xH$ of an element $x$ is equal to some right coset $Hy$ of a different element $y,$ since $x \in Hy$ as well, from a lemma from last week's lecture, $Hy = Hx,$ and so in fact the left coset and right coset of the \emph{same} element $x$ must also be equal. So it is sufficient to require that $xH = Hx.$}

\end{qq}

This question motivates the definition of \emph{normal subgroups.}
\begin{definition}
If $xH = Hx$ for each $x \in G,$ $H \subseteq G$ is called a \textbf{normal subgroup}.
Equivalently, the subgroup $H$ is normal if and only if it is invariant under conjugation by $x$; that is, $xHx^{-1} = H$. Using the notation from last lecture\footnote{The function $\varphi_x$ takes $g \mapsto xgx^{-1}.$}, a subgroup $H$ is normal if and only if $\varphi_x(H) = H$ for all $x \in G.$ 
\end{definition}

Let's look at some examples.
\begin{example}[Non-normal subgroup]
From above, the subgroup $\langle y \rangle$ is \emph{not} normal in $S_3.$ 

\end{example}

\begin{example}[Kernel]
Given a homomorphism $f: G \rightarrow G',$ the kernel of $f$ is \emph{always} normal. Take $k \in \ker(f).$ Then 

\[
f(xkx^{-1}) = f(x)f(k)f(x)^{-1} = f(k) = e_{G'},
\]
so $\varphi_x(\ker(f)) = \ker(f),$ and thus $\ker(f)$ is a normal subgroup. In fact, in future lectures, we will see that \emph{all} normal subgroups of a given group $G$ arise as the kernel of some homomorphism $f: G \rightarrow G'$ to a group $G'.$
\end{example}

\begin{example}
In $S_3$, the subgroup $\langle y \rangle $ is not normal, but $\langle x \rangle$ is normal. In particular, it is the kernel of the sign homomorphism $\text{sign}: S_3 \rightarrow \RR.$\footnote{A given permutation $\sigma$ can be written as a product of $i$ transpositions, where $i$ is unique up to parity. The sign homomorphism maps $\sigma$ to $(-1)^i.$} 
\end{example}

\subsection{The Correspondence Theorem}
Ealier in this lecture, we noticed that homomorphisms give us some information about subgroups. Can we make this more concrete?
\begin{qq}
Let $f$ be a homomorphism from $G$ to $G'$. Is there a relationship between the subgroups of $G$ and the subgroups of $G'$?

\[
\{\text{subgroups of $G$}\} \leftrightarrow \{\text{subgroups of $G'$}\}
\]
\end{qq}
\begin{ans}
In fact, we see that there is!
\begin{itemize}
    \item Given a subgroup of $G,$ a subgroup of $G'$ can be produced as follows. Let $f$ with the domain restricted to $H$ be denoted as $f|_H.$  Then a subgroup $H \leq G$ maps to $\im(f|_H) = f(H) \subseteq G',$ which is a subgroup of $G'.$
    
    \item Now, given $H' \leq G'$ and a subgroup of $G$ can be produced by taking the preimage 
    \[
    f^{-1}(H') = \{x \in G : f(x) \in H'\}.
    \]
    Is this subset of $G$ is actually a subgroup? It is! Let's just check that it's closed under composition. If $x, y \in f^{-1}(H),$ then $f(x), f(y) \in H'$, so $f(x)f(y) \in H'$, since $H'$ is closed under multiplication. Then $f(xy) \in H',$ so $xy \in f^{-1}(H).$ 
    
    If $H' = e_{G'},$ then its preimage is the kernel, and if $H' = G',$ then the preimage is all of $G.$ In general, the preimage is a subgroup somewhere in-between the kernel and the whole domain.
\end{itemize}
\end{ans}


Are these maps bijective, or inverses of each other? It can be easily seen that they are not; in particular, if $G$ is the trivial group and $G'$ is some more complicated group with many subgroups, every subgroup of $G'$ must always still map to the trivial group. It makes sense that these maps are not bijective, since $f$ is not an isomorphism, just an arbitrary homomorphism with no more restrictions. 

Two issues arise with these maps that make them non-bijective:
\begin{itemize}
    \item Any subgroup of $G$ must map to some subgroup of $G'$ that is contained within the image of $f,$ by construction, since $f(H) \subseteq \im(f).$ 
    \item The kernel $\ker(f) = f^{-1}(e_{G'}) \subseteq f^{-1}(H')$, so any subgroup not contained within the kernel cannot be mapped to by any subgroup of $G'.$
\end{itemize}

However, these are actually the only issues! If we are willing to put some restrictions on the homomorphism $f$ and the types of subgroups we look at, there \emph{is} actually a bijection between certain subgroups of $G$ and certain subgroups of $G'$.

In order to make things a little easier for now, we take a surjective homomorphism
$f: G \rightarrow G'$. The first issue then is no longer consequential, because the image is all of $G'$. Now, let's restrict the subgroups of $G$ to subgroups that contain $\ker(f).$ Then our maps (as described above) provide a bijection. 

\begin{theorem}[Correspondence Theorem]
For a surjective homomorphism $f$ with kernel $K,$ there is a bijective correspondence:

\[\{\text{subgroups of } G \text{ containing } K\} \leftrightarrow \{\text{subgroups of } G'\},\] 

where 
\begin{align*}
    \text{a subset of }G, H \supseteq K &\leadsto \text{its image } f(H) \leq G' \\
    H' \leq G' &\leadsto \text{its preimage } f^{-1}(H') \leq G.
\end{align*}
\end{theorem}

%Picture: %picture 1

\begin{example}[Roots of Unity]
Take
\begin{align*}
G = \CC^{*} &\xrightarrow{f} G' = \CC^{*}  \\
z &\mapsto z^2,
\end{align*}

which is a homomorphism because $G$ is abelian.

The kernel is $\ker(f) = \{\pm 1\}.$ We have a correspondence between $\RR^{\by} \leadsto \RR_{>0}$. 

For example, the eighth roots of unity correspond to the fourth roots of unity under this map. 

\[H = \{e^{\frac{2\pi ik}{8}}\} \leftrightsquigarrow H' = \{\pm 1, \pm i\}.\]
\end{example}

\newpage