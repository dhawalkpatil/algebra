%MIT OpenCourseWare: https://ocw.mit.edu
%RES.18-011 Algebra I Student Notes, Fall 2021
%License: Creative Commons BY-NC-SA 
%For information about citing these materials or our Terms of Use, visit: https://ocw.mit.edu/terms.

\section{Isomorphisms and Cosets}

\subsection{Review}
In the last lecture, we learned about subgroups and homomorphisms.  

\begin{definition}
We call $f: G \rightarrow G'$ a \textbf{homomorphism} if for all $a, b \in G,$ $f(a)f(b) = f(ab).$ 
\end{definition}
\begin{definition}
The \textbf{kernel} of a homomorphism $f$ is $\{a \in G : f(a) = e_{G'}\}$, and the \textbf{image} is the set of elements $b = f(a)$ for some $a.$
\end{definition}

The kernel and image of $f$ are subgroups of $G$ and $G'$, respectively. 

\subsection{Isomorphisms}

Homomorphisms are mappings between groups; now, we consider homomorphisms with additional constraints.
\begin{qq}
What information can we learn about groups using mappings between them?
\end{qq}
\begin{definition}
We call $f: G \rightarrow G'$ an \textbf{isomorphism} if $f$ is a bijective homomorphism. 
\end{definition}

In some sense, if there exists an isomorphism between two groups, they are the \emph{same} group; relabeling the elements of a group using an isomorphism and using the new product law yields the same products as before relabeling. Almost all the time, it is only necessary to consider groups \emph{up to isomorphism}.
\begin{example}
There exists an isomorphism $f: \ZZ_4 \rightarrow \langle i \rangle$ given by $n \mod 4 \mapsto i^n.$ In particular, we get 
\begin{align*}
    0 &\mapsto 1 \\
    1 &\mapsto i \\
    2 &\mapsto -1 \\
    3 &\mapsto -i.
\end{align*}
\end{example}

So the group generated by $i,$ which can be thought of as a rotation of the complex plane by $\pi/2$, is essentially "the same" as the integers modulo 4. 

\begin{example}
More generally, the group generated by $g, $ $\langle g \rangle = \{e, g,  g^2, \cdots, g^{d-1}\}$, where $d$ is the order of $g$, is isomorphic to $\mathbb{Z}_d = \{0, 1, \cdots, d-1\}.$ If the order of $g$ is infinite, then we have $\langle g\rangle \cong \ZZ.$
\end{example}

Here, the idea that an isomorphism is a "relabeling" of elements makes sense: since $g^{a}g^{b} = g^{a + b},$ relabeling $g^i$ with its exponent $i$ retains the important information in this situation. Thinking of $\langle g \rangle$ in this way yields precisely $\ZZ_d.$ 

\subsection{Automorphisms}

An important notion is that of an \emph{automorphism}, which is an isomorphism with more structure. 
\begin{definition}
An isomorphism from $G$ to $G$ is called an \textbf{automorphism.}
\end{definition}

If a homomorphism can be thought of as giving us some sort of "equivalence" between two groups, why do we care about automorphisms? We already \emph{have} an equivalence between $G$ and itself, namely the identity. The answer is that while the identity map $\id: G \rightarrow G$ is always an automorphism, more interesting ones exist as well! We can understand more about the symmetry and structure of a group using these automorphisms.

\begin{example}
A non-trivial automorphism from $\ZZ$ to itself is $f: \ZZ \rightarrow \ZZ$ taking $n \mapsto -n.$
\end{example}

From the existence of this nontrivial automorphism, we see that $\ZZ$ has a sort of "reflective" symmetry.\footnote{In particular, this automorphism $f$ corresponds to reflection of the number line across 0.}

\begin{example}[Inverse transpose]
Another non-trivial automorphism, on the set of invertible matrices, is the inverse transpose 
\begin{align*}
    f: GL_n(\RR) &\rightarrow GL_n(\RR) \\
    A &\mapsto (A^t)^{-1}
\end{align*}
\end{example}

Many other automorphisms exist for $GL_n(\RR),$\footnote{For example, just the transpose or just the inverse are automorphisms, and in fact they are commuting automorphisms, since the transpose and inverse can be taken in either order.} since it is a group with lots of structure and symmetry. 

\begin{example}[Conjugation]
A very important automorphism is \textbf{conjugation} by a fixed element $a \in G.$ We let $\phi_a: G \rightarrow G$ be such that
\[
\phi_a(x) = axa^{-1}.
\]

We can check the conditions to show that conjugation by $a$ is an automorphism:
\begin{itemize}
    \item \textbf{Homomorphism.} \[
    \phi_a(x)\phi_a(y) = axa^{-1}aya^{-1} = axya^{-1} = \phi_a(xy).
    \]
    \item \textbf{Bijection.} We have an inverse function $\phi_{a^{-1}}$: 
    \[
    \phi_{a^{-1}} \circ \phi_{a} = \phi_{a} \circ \phi_{a^{-1}} = \text{id}.
    \]
\end{itemize}
\end{example} 

Note that if $G$ is abelian, then $\phi_a = \id.$

Any automorphism that can be obtained by conjugation is called an \textbf{inner automorphism}; any group intrinsically has inner automorphisms coming from conjugation by each of the elements (we can always find these automorphisms to work with). Some groups also have \textbf{outer automorphisms}, which are what we call any automorphisms that are not inner. For example, on the integers, the only inner automorphism is the identity function, since they are abelian.\footnote{For an abelian group, $axa^{-1} = aa^{-1}x = x$.}

\subsection{Cosets}

Throughout this section, we use the notation $K \coloneqq \ker(f)$.

\begin{qq}
When do two elements of $G$ get mapped to the same element of $G'$? When does $f(a) = f(b) \in G'$?

\end{qq}

Given a subgroup of $G,$ we can find "copies" of the subgroup inside $G.$
\begin{definition}
Given $H \subseteq G$ a subgroup, a \textbf{left coset} of $H$ is a subset of the form \[aH \coloneqq \{ax : x \in H\}\] for some $a \in G$.
\end{definition}

Let's start with a couple of examples. 

\begin{example}[Cosets in $S_3$]
Let's use our favorite non-abelian group, $G = S_3 = \langle (123), (12) \rangle = \langle x, y \rangle,$ and let our subgroup $H$ be $\{e, y\}.$ Then 
\[
eH = H = \{e, y\} = yH;
\]
\[
xH = \{x, xy\} = xyH;
\]
and
\[
x^2H = \{x^2, x^2y\} = x^2yH.
\]

We have three different cosets, since we can get each coset one of two ways.
\end{example}

\begin{example}
If we let $G = \ZZ$ and $H = 2\ZZ,$ we get 
\[
0 + H = 2\ZZ = \text{evens} = 2 + H = \cdots,
\]
and 
\[
1 + H = 1 + 2\ZZ = \text{odd integers} = 3 + H = \cdots .
\]
\end{example}

In this example, the odd integers are like a "copy" of the even integers, shifted over by 1. From these examples, we notice a couple of properties about cosets of a given subgroup.

\begin{proposition}
All cosets of $H$ have the same order as $H.$%\footnote{As demonstrated in the preceding finite examples.}
\end{proposition}
\begin{proof}
We can prove this by taking the function $f_a: H \rightarrow aH$ which maps $h \mapsto ah.$ This is a bijection because it is invertible; the inverse is $f_{a^{-1}}.$\footnote{I can undo any $f_a$ in a \textbf{unique} way by multiplying again on the left by $a^{-1}$. This is something that breaks down with monoids or semigroups or other more complicated structures.}
\end{proof}

\begin{proposition}
Cosets of $H$ form a \textbf{partition} of the group $G$.\footnote{A partition of a set $S$ is a subdivision of $S$ into disjoint subsets.}
\end{proposition}


To prove this, we use the following lemma.
\begin{lemma}\label{coset lemma 1}
Given a coset $C \subset G$ of $H,$ take $b \in C.$ Then, $C = bH.$

\end{lemma}
\begin{proof}
If $C$ is a coset, then $C = aH$ for some $a \in G.$ If $b \in C,$ then $b = ah$ for some $h \in H,$ and $a = bh^{-1}.$ Then 
\[
bH = \{bh' : h' \in H\} = \{ahh'| h' \in H\} \subseteq aH.
\]

Using $a = bh^{-1},$ we can similarly show that $aH \subseteq bH,$ and so $aH = bH.$\footnote{So for a given coset $C$, we can use any of the elements in it as the representative $a$ such that $C = aH.$}
\end{proof}
\begin{proof}
Now, we prove our proposition.
\begin{itemize}
    \item Every $x \in G$ is in some coset. Take $C = xH.$ Then $x \in C.$
    \item Cosets are disjoint. If not, let $C, C'$ be distinct cosets, and take $y$ in their intersection. Then $yH = C$ and $yH = C'$ by Lemma \ref{coset lemma 1}, and so $C = C'.$
\end{itemize}
\end{proof}

With this conception of \emph{cosets}, we have the answer to our question:

\begin{ans}
If $f(a) = f(b)$, then $f(a)^{-1} f(b) = e_{G'}.$ In particular, $f(a^{-1}b) = e_{G'},$ so $a^{-1}b \in K$, the kernel of $f.$ Then, we have that $b \in aK,$ or $b = ak$ where $f(k) = e_{G'}.$ So $f(a) = f(b)$ if $a$ is in the same left coset of the kernel as $b.$

\end{ans}

\subsection{Lagrange's Theorem}

In fact, thinking about cosets gives us quite a restrictive result on subgroups, known as Lagrange's Theorem.
\begin{qq}
What information do we automatically have about subgroups of a given group?
\end{qq}
\begin{definition}
The \textbf{index} of $H \subseteq G$ is $[G: H],$ the number of left cosets. 
\end{definition}

\begin{theorem}
We have
\[
|G| = [G:H]|H|.
\]
\end{theorem}

\begin{proof}
This is true because each of the cosets have the same number of elements and partition $G$. 

So we have
\[
|G| = \sum_{\text{left cosets } C} |C| = \sum_{\text{left cosets } C} |H| = [G:H]|H|.
\]

That is, the order of $G$ is the number of left cosets multiplied by the number of elements in each one (which is just $|H|$).
\end{proof}
\begin{example}
For $S_3$, we have $6 = 3 \cdot 2.$
\end{example}

From our theorem, we get Lagrange's Theorem:
\begin{corollary}[Lagrange's Theorem.]
For $H$ a subgroup of $G,$ $|H|$ is a divisor of $|G|.$
\end{corollary}

We have an important corollary about the structure of cyclic groups.
\begin{corollary}
If $|G|$ is a prime $p,$ then $G$ is a cyclic group. 
\end{corollary}
\begin{proof}
Pick $x \neq e \in G.$ Then $\langle x \rangle \subseteq G$. Since the order of $x$ cannot be 1, since it is not the identity, the order of $x$ has to be $p,$ since $p$ is prime. Therefore, $\langle x \rangle = G,$ and so $G$ is cyclic, generated by $x.$
\end{proof}

In general, for $x \in G,$ the order of $x$ is the size of $\langle x \rangle,$ which divides $G.$ So the order of any element divides the size of the group. 

\newpage