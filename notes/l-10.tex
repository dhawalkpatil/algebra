\noindent\emph{Lecture - 11 \hfill 26 Sep 2024}

Firs Mid Term Exam on 03 Oct 24, Thu.
05 questions will be asked and the test will be for 75 minutes.
Syllabus will include Group Actions, Sylow Theorems, Semidirect product.

Relationship between action and automorphisms.


We ask when a semidirect product is a direct product.

$H \rtimes K = H \times K$ iff
$ \phi $ is trivial iff $ \phi_k = \id_H$ for each $k \in K$
iff $1 \times  K \triangleleft H \rtimes_ \phi K.$


\begin{theorem}
	In $H \rtimes _ \phi K,$ the subgroup $1 \times  K \triangleleft H \rtimes _ \phi K$
	iff $ \phi \colon K \to \Aut(H)$ is trivial.	
\end{theorem}

\begin{proof}
	Recall that $ H \rtimes _ \phi K $ is generated by $H \times  1$ and $ 1 \times  K.$
	Also, $ 1 \times  K$ is normal in $H \rtimes_ \phi$ iff $ (h, 1 ) ( 1,k) ( h^{-1}, 1)^{-1}$ is in $ t \times K$ for all $ h in H$ and $ k in K.$ Namely,
	$(h,k)(h^{-1},k) = (h \phi_k(h^{-1}), k) $ is in $ 1 \times  K.$
	The latter is possible if and only if $ h \phi_k(h^{-1}) = 1,$ that is $\phi_k(h) = h$
	for each $ h$ in $H$ and $k$ in $K.$ That means, $\phi_k = \id_H$ for each $k$ in $K.$
\end{proof}

\begin{example}[Affine transformations as a semidirect product]
	Let $ H = (\mathbb{R}, +), K = (\mathbb{R}^\ast, \cdot)$ and $ \phi \colon \mathbb{R}
	^\ast \to \Aut(\mathbb{R})$ defiend by $ \phi(x) = \phi_x$ where $ \phi_x(y) = xy$
	for all $y$ in $\mathbb{R}.$
	Then $\mathbb{R} \rtimes_ \phi \mathbb{R}$ has operation 
	$(a,b)(a',b') = )a + \phi_b(a'), bb') = (a + ba',bb').$
	We have $$\mathbb{R} \rtimes _\phi \mathbb{R} \equiv \operatorname{Aff}(\mathbb{R})
	 = \left\{ 
	 \begin{pmatrix}
	  b & a \\ 0 & 1 \end{pmatrix} \; : \; a \in \mathbb{R}, b \in \mathbb{R}^\ast 
  \right\} .$$
\end{example}

\begin{theorem}[Semidirect Product Recognition]
	Let $G$ be a group with subgroups $H$ and $K$ satisfying 
	\begin{enumerate}
		\item $ G = HK, $ 
		\item $ H \cap K = \{ 1_G \} $ 
		\item $ H \triangleleft G .$ 
	\end{enumerate}
\end{theorem}


Let $ \phi \colon K \Aut(H)$ be defind where $\phi_k$ acts on $H$ by conjugation
for each $k$ in $K$ with $ \phi(k)(h) = k h^{-1} k$
for $h$ in $H.$ Then $ \phi$ is a group homomorphism and the mapping $ f \colon H \rtimes_
\phi K \to G$ with $f(h,k) = hk$ is an isomorphism.
To check that $\phi$ is a  homomorphism, we take $x,y$ in $K$ and check that
$$ \phi_{xy} (h) = y x h x^{-1} y^{-1} = \phi_y(x h x^{-1}) = \phi_y( \phi_x(h))
= \phi_y \circ \phi_x (h)$$
for  any $h$ in $H.$
Given, $h, h'$ in $H$ and $k, k'$ in $K,$ we check that $f$ is a homomorphism:
$$f((h,k)(h',k')) = f(h \phi_k(h'), kk') = h \phi_k(h') kk'
=  h k h k^{-1} k k' = hk h'k'.$$

\begin{example}[Permutation group $S_n$ as a semidirect product]
	Let $H = S_n, G = S_n$ for some integer $n \geq 3,$ and $ K = \left<(1,2) \right>.$
	We have $ G = H \cup H (1,2) = HK,$ $H \cap K = \{ 1_G \} = \{ \id \},$
	$H \triangleleft G$ because it has index 2.
	We can then express $G = H \rtimes_ \phi K,$ where $ \phi_k(h) = khk^{-1}.$
\end{example}

\begin{example}
	In $G = S_4,$ with $ \lvert G \rvert = 24 = 2^3 \times  3,$ $H \in Syl_2(G),$
	$K \in Syl_3(G).$ Then $H$ has order $8$ and some elements of order 2 and 4.
	So $H \equiv D_4,$ $K \equiv Z_3.$ $G$ is not a semidirect product of $H$ and $K.$
	We have $G =HK$ and $ H \cap K = \{ 1_G \}$ but $H \not \triangleleft G$ and
	$K \not \triangleleft G.$
\end{example}

\begin{example}[Groups of order $pq$]
	We know that if $G$ is a group of order $p,q$ where $p,q$ are prime integers and $p < q,$
	then either $G$ is abelian with $G \equiv \mathbb{Z}_{pq}$ or $G$ is not abelian and
	$p | q-1.$
	As $q > p,$ there exists an element $b$ of order $q$ WHY?????
	So, there is a group $\left< b \right>$ of order $q.$ This group is normal because
	if $a^{-1} \left<b \right> a \not = \left< b \right>$ for some $a$ in $g,$
	then the product of these groups has order $q^2$ and is contained in $G.$
	As this is not possible, we get that $ \left<b \right> \triangleleft G.$
	Thus $b^{-1}ab = a^d$ for some positive integer $d.$ Then, we have
	$ a = b^{-p} a b^p = a^{d^p}$ implying that $e = a^{d^p -1} $
\end{example}

Let $P \in Syl_p(G)$ and $Q \in Syl_q(G).$ We know that $Q \triangleleft G$ because WHY????
Then $G = PQ$ and $ P \cap Q = \{ 1_G \}.$ So that $G = Q \rtimes_\phi P$ for some 
group homomorphism $ \phi \colon P \to \Aut(Q).$ The automorphism group of a cyclic group
of order $n$ is a cyclic group of order REALLY??? equal to the number of integers coprime 
to and less than $n.$ So, $ \lvert \Aut(Q) \rvert = q-1.$ If $ p\not | q-1,$ then $
\lvert \phi(P) \rvert = 1$ and so $G$ is abelian. This is because of
the fact that $\phi(P) \leq \Aut(Q)$ implies that $ \lvert \phi \rvert = \frac{ \lvert P
\rvert}{ \lvert \ker(\phi) \rvert}$ is either $1$ or $p.$
If $p | q-1,$ let $P = \left<y \right>$ and $\left< \gamma \right>$ be the unique order $p$
subgroup of $\Aut(Q)$ and $ \phi \colon P \to \Aut(Q)$ defined by $\phi(y) = r.$
We have $\phi(P) = \left< \gamma \right>$ for some $ \gamma$ in $\Aut(Q)$ because the
latter is a cyclic group. If $1 \leq i \leq p-1,$ G = \ltimes_ \phi Q.$


