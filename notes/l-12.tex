
\noindent
\emph{Lecture - 11 \hfill 01 Oct 24, Tue}

Today we will discuss the Schur Zassenhaus Theorem.
Suppose $N$ is a normal subgroup of $G.$ Can we construct $G$
from $N$ and $G / N?$
Say $G \equiv \mathbb{Z}_{p^2},$ and $N \triangleleft G,$ with
$N \equiv \mathbb{Z}_{p}.$ Then $G / n \equiv \mathbb{Z}_{p},$
and constructing $G$ from $N$ and $G/N$ is not straightforward.
\begin{definition}[Complement]
	Let $G$ be a group and $N \triangleleft G.$ We say that
	a subgroup $H$ of $G$ is complement of $N$ if $G = N \rtimes H.$
\end{definition}

Not every normal subgroup has a complement. For instance,
$G = S_4$ has a normal subroup $H \equiv D_4$ which is the unique
Sylow~$2$\nobreakdash-subgroup. The unique REALLY???
Sylow~$3$\nobreakdash-subgroup $C_3$ is not complement to $D_4.$


\begin{definition}[Hall Subgroup]
	A subgroup $H$ of a finite group $G$ is a Hall subgroup if 
	$( \lvert H \rvert, [ G \, : \, H ] ) = 1.$
\end{definition}
For example, every Sylow~$p$\nobreakdash-subgroup is a Hall subgroup.

\begin{theorem}[Schur Zassenhaus]
	Any normal Hall subgroup $H$ of a finite group $G$ has a 
	complement and all the complements are conjugate in $G.$
\end{theorem}
\begin{example}
	Let $p$ be an odd prime and $G$ be a group of order $2p.$
	There exists a subgroup $H$ of $G$ of order $p.$ $H$ is
	normal in $G$ and also $( \lvert H \rvert, [G \, : \, H])
	= 1$ so $H$ is a Hall subgroup of $G.$ Fill in the details.
	why are all elements of order 2 conjugate to each other?
\end{example}

\begin{lemma}[Frattini argument]
	Let $G$ be a finite group, $N \triangleleft G,$ and $P$
	be a Sylow~$p$\nobreakdash-subgroup of $N.$ Then,
	$G = N_G(P) N.$
\end{lemma}
\begin{proof}
	Let $g$ be in $G.$ Then $ g^{-1} P g \leq g^{-1} N g = N.$
	So, the action of $G$ on $Syl_p(N)$ via conjugation is well-%
	defined. This is because $P$ is conjugate to each Sylow~$p$%
	\nobreakdash-subgroup of $N$ by Sylow Conjugacy Theorem.
	For each $g$ in $G,$ $g^{-1}Pg$ is in $Syl_p(N),$ so there
	exists $n$ in $N$ such that $n^{-1} g^{-1} P g n = P.$ Then $gn$ 
	is in $N_G(P).$ So $g \in N_G(P) n^{-1} \subseteq N_G(P) N.$
	Thus $G \subseteq N_G(P) N.$ This proves the required equality.
\end{proof}


\begin{proof}[Proof of Schur Zassenhaus Theorem]
	Let $N$ be a normal Hall subgroup of a finite group $G$ with
	$ n = [G \, : \, N].$ This means $\GCD( \lvert N \rvert, n) = 1.$
	For this GCD to be well defined, $G$ has to be finite.
	
	\emph{Step 1 }\quad It suffices to show that $G$ has a 
	subgroup $K$ of order $n.$

	Note that $NK = 
	\frac{ \lvert N \rvert \lvert K \rvert}{ \lvert N \cap K \rvert}
	 = \lvert N \rvert \lvert K \rvert = \lvert G \rvert$
	because $N \cap K = \{ 1_G \}$ because the order of this group
	divides the order of $N$ as well as of $K.$
	By recognition of semidirect product theorem, we know that
	$G = N \rtimes K.$

	We proceed to prove by induction. The induction hypothesis
	is that for each groupwith order less than $ \lvert G \rvert,$
	which has a normal Hall subgroup $H'$ also has a subgroup
	with order equal to the index of $H'.$

	\emph{Step 2} \quad Let $P$ be a Sylow subgroup of $N.$
	We can assume that $P$ is normal in $G,$ for otherwise, 
	we can find a subgroup in $G$ of order $n.$

	By Frattini argument,
	$ G = N_G(P) N.$ Observe that $ N_N(P) = N_G(P) \cap N $ 
	is a normal subgroup of $N_G(P) $ because $N$
	is normal in $G.$ So we can apply the Second Isomorphism Theorem
	to conclude that
	$$ \frac{G}{N} \cong \frac{N_G(P) N}{N} \cong
	\frac{N_G(P)}{N_G(P) \cap N} = \frac{N_G(P)}{ N_N(P)} $$
	Thus $ n = [G \, : \, N] = [ N_G(P) \, : \, N_N(P) ].$ If
	$N_G(P) < G,$ that is $ P \not \triangleleft G,$ then we shall
	apply the induction hypothesis to $N_G(P).$ We know that
	$N_N(P) \triangleleft N_G(P) $ and it is also a Hall subgroup
	of $G$ because $ \lvert N_N(P) \rvert \, | \, \lvert N \rvert$ by
	Lagrange's Theorem. We know
	$ [ N_G(P) \, : \, N_N(P) ] = n$ and $ GCD( \lvert N \rvert,
	n ) = 1$ so $\GCD( \lvert N_N(P) \rvert, n) = 1.$
	Thus, applying the induction hypothesis, we get that
	$N_G(P)$ contains a subgroup of order $[ N_G(P) \, : \, 
	N_N(P)] = n.$ Thus $G$ contains a subgroup of order $n.$

	\emph{Step 3} \quad Proceeding with the case that $P$ is normal in $G,$ we now show that we can assume that $N = P.$

	If $ P < N,$ then 
	$N /P \triangleleft G / P$ and $ [ G/P \, : \, N /P] =
	[ G \, : \, N ] =n,$
	by correspondence theorem and $ \GCD( \lvert N/P \rvert, n) = 1$ because 
	$\GCD( \lvert N \rvert, n) = 1.$ So we now get that
	$N/ P$ is a normal Hall subgroup of $G/P.$ By Correspondence
	Theorem, this group must be of the form $L/P$ where $L$ is some
	subgroup of $G$ containing $P.$ If we can show that $ L < G,$
	then we can find a subgroup of $G$ of order $n$ by 
	applyign the induction hypothesis to $L.$
	To show this, consider 
	$L \cap N.$ Its order divides $ \lvert N \rvert$ as well
	as $ \lvert L \rvert = n \lvert P \rvert.$ As $\GCD(n, \lvert 
	N \rvert) = 1,$ we get that $ \lvert L \cap N \rvert $ divides
	$ \lvert P \rvert.$ So $ \lvert L \cap N \rvert \leq \lvert P \rvert.$
	On the other hand, we have $ \lvert P \rvert \leq
	\lvert L \cap N \rvert$ because $P$ is contained in $L$ as well
	as $N.$ So, $P = L \cap N.$

	If $P <N,$ then
	there exists $x$ in $ N \setminus P.$ So $x$ is in $N$ but not
	in $L.$ This shows that $L < G.$

	An alternative way to view this is to realise that if $P$
	is not $N,$ then there is an element $xP$ in $N/P$ which is not
	$P.$ As $o(xP)$ would divide $ \lvert N /P \rvert$ and 
	$\GCD( \lvert N/P \rvert, \lvert L/P \rvert),$ we get that
	$xP$ is not in $L/P.$ So, we get that $L < G.$
	
	For the next step, we assume that $N=P$ in addition to the 
	assumption that $N \triangleleft G.$

	\emph{Step 4} \quad We assume for this case that $N$ is not
	abelian and prove that in this case there exists a subgroup of
	order $n$ in $G.$

	Let $N$ be non abelian and denote its center as $Z = Z(N).$
	So, $ 1 \leq Z \triangleleft N.$ Since $Z$ is a characteristic
	subgroup of $N$ and $N \triangleleft G,$ we have $Z \triangleleft G.$ Observing that $ N/Z \triangleleft G/Z,$ and 
	$[G/Z \,:\, N / Z] = [G \,:\, N] = n,$ we apply the induction
	hypothesis to $G/Z$ whose order is less than $ \lvert G \rvert$
	because the centre of $N$ is nontrivial. So, $G/Z$ has
	a subgroup $J/Z$ of order $n.$

	We show that $J<G$ and then use the induction hypothesis to 
	conclude that there is a subgroup of order $n$ contained in
	$J$ and thus in $G.$ As $\GCD( \lvert J/Z \rvert, \lvert N
	/Z \rvert ) = 1,$ we get that $ (J \cap N) / Z = 
	J/Z \cap N/Z$ is the trivial subgroup of $G/Z.$ 
	So, $J \cap N = Z.$ There exists $x$ in $N \setminus Z$
	because $N$ is not abelian. We get that $x$ is in $N$ and 
	thus in $G$ but not in $J$
	because $x$ is not in $Z = J \cap N.$
	
	WHY IS THE CENTER NONTRIVIAL?
\end{proof}

\begin{corollary}
	Fix a prime integer $p.$ For a finite group $G$ with order divisible by $p,$ the following statements are equivalent
	\begin{enumerate}
		\item $ \lvert \Aut(G) \rvert $ is not divisible by $p.$
		\item $G \cong \mathbb{Z}_p \times H,$ where
			$p$ does not divide $ \lvert H \rvert$ or
			$ \lvert \Aut(H) \rvert.$
	\end{enumerate}
\end{corollary}

\begin{proof}
	$(2 \implies 1)	$ For $G,$ as given, we have
	$ \Aut(G) \cong \Aut(\mathbb{Z}_p) \times \Aut(H)
	\cong \mathbb{Z}_p^\ast \times \Aut(H).$
	As $p$ does not divide $ \lvert \mathbb{Z}_p^\ast \rvert = p-1,$
	we see that $p$ does not divide $ \Aut(H).$

	$ (1 \implies 2) $ Let $P \in Syl_p(G),$ we aim to show that $G = 
	P \times  H.$
\end{proof}

