\noindent
\emph{Lecture 09 \hfill 24 Sep 24, Tue}
\section{Semidirect Product}

\begin{proposition}[Number of Representations]\label{prop:number-of-representations}
	Let $H$ and $K$ be subgroups of a group $G.$ The number of distinct ways of
	writing any given element in $HK$ in the form $hk,$ where $h$ is in $H$ and $k$ 
	is in $K,$ is $ \lvert H \cap K. \rvert.$
\end{proposition}

Direct Product
Suppose $H$ and $K$ are groups embedded in $H \times K$ in the standard way
$$ H \to H \times  K \qquad h \mapsto (h,1) $$
and 
$$ K \to H \times  K \qquad k \mapsto (1,k) .$$
$1$ in the first, respectively second, coordinate represents the identity in $H,$
respectively $K.$
Then the following properties hold
\begin{enumerate}
	\item $H \times  1$ and $1 \times  K$ generate $ H \times  K.$
		For $(h, k )$ in $ H \cap K,$ we have $(h,k) = (h, 1) (1, k).$
	\item $ ( H \times 1) \cap (1 \times  K) = \{ ( 1, 1 ) \}.$
	\item Commutativty:
		$$(h,1) ( 1,k) = (1,k) (h,1).$$
\end{enumerate}

\begin{theorem}[Direct Product Recognition]
	Let $G$ be a group with subgroups $H$ and $K,$ where
	\begin{enumerate}
		\item $G = H K.$
		\item $H \cap K = \{ 1 \}.$
		\item $H$ lies in the centre of $K,$ or, equivalently,
			$K$ lies in the centre of $H.$
	\end{enumerate}
	Then the map $ H \times K \to G$ defined by $(h, k) \mapsto hk$
	is an isomorphism. In other words, $G$ is the direct product of
	$H$ and $K.$
\end{theorem}

\begin{example}
	Let $I$ be an $m$\nobreakdash-subset of $\{ 1, 2, \dotsc, m \}$ and
	let $G$ be the setwise stabiliser of $I$ in $S_n,$
	$$ G = \left\{ \sigma \in S_n \; \mid\vert \; \sigma(I) = I \right\} .$$
	Write $J = \{ 1, 2, \dotsc, n  \} \setminus I.$ G is also the setwise 
	stabiliser of $J.$ Suppose 
	$$ H = \{ \sigma \in G \; \vert \; \sigma(i) = i, \quad \forall i \in I \} $$
	and
	$$ K = \{ \sigma \in G \; \vert \; \sigma(i) = i , \quad \forall i \in J \} .$$
	be the pointwise stabiliser of $H.$ We know that $H \triangleleft G$ and 
	$K \triangleleft G.$ Moreover, $H \cap K = \{ 1_G \}.$ Then $HK \cong H \times  K.$
	Let $h \in H$ and $k \in K.$ We want to prove that $hk = kh.$
	We have $ h \in Sym( \{ 1, 2, \dotsc, n \} \setminus I ) = Sym(J)$
	and $k \in Sym( \{ 1, 2, \dotsc, n \} \setminus J) = Sym(I).$
	For each $\sigma$ in $G$ stabilising $I$ and $J,$ we can write
	$\sigma = \sigma_I \sigma_J,$ where $\sigma_I \in Sym(J), \sigma_J \in Sym(I).$
\end{example}


\begin{example}
	Let $G=  S_n$ and $n \geq 3.$ Let $H =  A_n$ and $ K = \langle ( 1 \; 2 ) \rangle .$
	We have $H \triangleleft G, K \leq G$ and $ H \cap K = \{ 1_G \}.$ However 
	as we know that $H$ is not normal in $G,$ we realise that $G$ is not the direct
	product of $H$ and $K.$
\end{example}

Semidirect product
Let $H$ and $K$ be subgroups of $G.$ Suppose $H \triangleleft G,$ then $HK \leq G.$
For $hk$ and $h'k'$ in $hk,$ we have
$$ (hk)(h'k') = hkh' k^{-1} k k' = h (k h' k^{-1}) kk'$$
which is in $HK$ because $k h k^{-1}$ is in $H.$
Also, $(hk)^{-1} = k^{-1} h^{-1} = k^{-1} h^{-1} k k^{-1} .$
$K$ acts on $H$ via conjugaction. Since $ H \triangleleft G,$ this conjugation
is an automorphism of $K$ on $H.$

Consider two groups $H$ and $K$ not necessarily related. Suppose there is a group 
homomorphism $\; \phi \colon K \to \operatorname{Aut}(H) \;$ is an action of $K$ on $H$ via $ \phi.$
$ \phi_k \mapsto \phi_k \in \operatorname{Aut}(H) .$
Since $ \phi$ is a group homomorphism, $ \phi_{k_1} \circ \phi_{k_2} = \phi_{k_1 k_2}$
and $ \phi_{k}^{-1} = \phi_{k^{-1}}.$


\begin{definition}[Semidirect Product]
	For two groups $H$ nad $K$ and an action (a group homomorphism) $\phi \colon K \to 
	\operatorname{Aut}(H) .$ The corresponding semidirect product $H \rtimes_\phi K$
	is defined as follows
	\begin{enumerate}
		\item as a set $H \rtimes_\phi k = \{ (hk) \; | \; h \in H, k \in K\}$
		\item operation $(h,k) (h',k') = (h \phi_k(h') k k')$
	\end{enumerate}
\end{definition}

If $H, K \leq G$ and $ \phi_k(h') = k h' k^{-1}.$ To show that $H \rtimes_\phi K$
is a subgroup, we need to verifiy that
the operation is closed, $(1_H, 1_K) $ is the identity,
the inverse of $h,k$ is $(\phi_{k^{-1}} (h^{-1} ), k^{-1} )$
and the operation is associative.

\begin{example}
	Let $H = \{ \pm 1 \}$ be a multiplicative group of two elements acting on
	$\mathbb{Z},$ the additive group of all integers by automorphisms. Then 
	there is a homomorphism $ \phi \colon \{ \pm 1 \} \mapsto \operatorname{Aut}
	(\mathbb{Z}).$ We write $\phi(1) = \phi_1)$ where $ \phi_1(n) = n$ 
	for all $n$ in $\mathbb{Z}$ and $ \phi(-1) = \phi_{-1}$ where
	$ \phi_{-1} (n) = -n$ for all $n$ in $\mathbb{Z}.$
	Then the semidirect product $ \mathbb{Z} \rtimes_{ \phi} \{ \pm 1 \}$ with
	operatiions defined as
	$$ (a, \epsilon) (a', \epsilon')=  (1 + \phi_\epsilon(a'), \epsilon \epsilon')
	= (a + \epsilon a', \epsilon \epsilon')$$
	where the operation in the first coordinate is addition in $\mathbb{Z}$ and in the
	second coordinate is multiplication.
\end{example}

Moreover for any group homomorphism $ \Theta \colon \mathbb{Z} \to \operatorname{Z}$
given by $\Theta_n \in \operatorname{Aut}(\mathbb{Z}),$ the semidirect product
$ \mathbb{Z} \rtimes_\Theta \mathbb{Z}$ with operation
$$ (n,m)(n',m') = (n + \Theta_m(n'), m + m')
= (n + (-1)^m n', m + m') .$$


The relation between $H K$ and $ \rtimes_\phi K.$

\begin{theorem}[Constructing Semidirect Products]
	Inside $H \rtimes_\phi K,$ we have
	$$ H \equiv H \times  1 = \{ (h,1) \; | \; h \in H \} $$
	by $ h \mapsto (h,1) $
	and 
	$$ K \equiv 1 \times  K = \{ (1, k) \; | \; k \in K \} $$
	for each $ h \in H $ and $k \in K$
	$$(h,k) = (h, 1) (1, k) = (1,k) (\phi^{-1} (h), 1) .$$
	$H \rtimes_\phi K$ is generated by $ H \times 1$ and $1 \times K.$
	$H \times 1$ is a normal subgroup of $H \rtimes_\phi K$ with
	conjugation
	$$(1,k)(h,1)(1,k)^{-1}= (\phi_k(h), 1) \in H \times  1.$$
	In particular, for every $h$ in $H$ and $k$ in $K$,
	$(h,1)$ and $(1,k)$ commutative iff $ \phi \colon K \to \operatorname{Aut}(H)$
	is the trivial action where $\phi_k = \phi(k)$ is the identity mapping for each $k$
	in $K.$
\end{theorem}
