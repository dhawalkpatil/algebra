\section{Sylow's Theorems}
The Lagrange's Theorem states that if $G$ is a finite group and $H$ is a subgroup of $G,$
then $ \lvert H \rvert,$ the order of $H$ divides $ \lvert G \rvert,$ the order of $G.$
We explore the following question: If $G$ is a finite group, and $k$ is an integer
which divides $ \lvert G \rvert ,$ does there exist a subgroup $H$ of order $k?$
This is, in a way, the converse of Lagrange's Theorem.

By the Theorem on structure of finite abelian groups, we know that this is indeed true
for finite abelian groups.
However, it fails for non abelian groups. For example, $ \lvert A_{4} \rvert = 12,$
has no subgroup of order 6.

\begin{lemma}[Binomial Coefficients modula a prime]
	Let $ n = p^{\alpha} m$ where $p$ is a prime.
	Then $ \binom{n}{p^{\alpha}} \equiv m (\mod p).$
\end{lemma}

\begin{proof}
	From Fermate-Euler's Theorem, we know that for any positive integer
	$x,$
	$ (x+1)^{p ^\alpha} \equiv x^{p^\alpha} + 1 (\mod p) $	
	Applying this to $x^m $ instead of $x,$ we get

	$ (x+1)^{p ^\alpha m} \equiv x^{p^\alpha m} + 1 (\mod p) .$

\end{proof}

\begin{theorem}[Sylow's Existence Theorem]
Let $G$ be a finite group of order $p^{\alpha}m,$ where $p$ is prime and $p \not |m.$
Then $G$ has a subgroup $H$of order $p^{\alpha}.$
\end{theorem}

\begin{proof}
	Let $\Omega$ be the collection of all subsets $X$ of $G$ of size $p^\alpha.$
	Then,
	$$ \lvert \Omega \rvert = \binom{ \lvert G \rvert }{p^\alpha}
	= \binom{p^\alpha m} {p^\alpha }
	\equiv m \not \equiv 0 ( \mod p).$$
	Let $G$ act on $ \Omega $ via right regular action.
	Since $p \not  | \, \lvert \Omega \rvert ,$ there exists an orbit 
	$O$ such that $p \not |\, \lvert O \rvert,$
	for the orbits of the action partition $ \Omega. $

	Let $X_o \in O$ and $H $ be $ G _{X_o} \leq G,$ the stabiliser of $X_o.$
	By the orbit stabiliser theorem, the size of this orbit is
	$ \lvert O \rvert =  \frac{ \lvert G  \rvert }{ \lvert H \rvert },$
	the number of cosets of $H.$
	Since $p \not | \, \lvert O \rvert$
	but $p^\alpha \, | \, \lvert G \rvert = \lvert H \rvert \lvert O \rvert,$
	we must have $p^{\alpha} | \, \lvert H \rvert.$
	For each $ x \in X_o$ and $ h \in H,$ we have $ xh \in X_o.$
	So, we have $x H \subset X_o.$
	Then,
	$ \lvert H \rvert = \lvert x H \rvert \leq \lvert X_o \rvert = p^\alpha.$
	As $H \leq G,$ we get $ \lvert H \rvert = p^\alpha.$

\end{proof}

\begin{definition}[Sylow $p-$subgroup]
	Let $G$ be a finite group and $p$ be a prime integer.
	A Sylow $p$\nobreakdash-subgroup of $G$ is a subgroup
	$P \leq G$ such that $ \lvert P \rvert = p^\alpha \, | \, \lvert G \rvert$
	but $ \lvert p^{\alpha +1}  \rvert \, \not | \, \lvert G \rvert .$
	The set of all Sylow $p$\nobreakdash-subgroups of $G$ is denoted 
	by $Syl_p(G).$
\end{definition}

For each prime integer $p | \lvert G \rvert,$ $Syl_p(G) \not = \left\{ \phi \right\} .$
For prime integers $p$ such that $p \not | \, \lvert G \rvert,$
set $Syl_p(G) = \left\{ \left\{ 1_G \right\}  \right\}.$

\begin{corollary}[Cauchy's Theorem]
	Let $G$ be a finite group with $p | \lvert G \rvert$ where $p$
	is some prime integer. Then, $G$ has an element of order $p.$
\end{corollary}

\begin{proof}
	By Sylow's Existence theorem, there exists a Sylow $p$\nobreakdash- subgroup
	$H$ of $G.$
	Since $p | \lvert G  \rvert,$
	$P$ is a nontrivial subgroup of $G.$
	That means there exists $x \in P \setminus \left\{ 1_G \right\}$
	having order $o(x) = p^e,$ where $1 \leq e.$
	Set $y = x^{p^{e-1}}.$ Then, $o(y) = p.$	
\end{proof}

Each $P \in Syl_p(G)$ is a maximal subgroup of $G.$

\begin{theorem}[Sylow Development]
	Let $G$ be a finite group and $P \leq G$ be a $p$\nobreakdash-subgroup,
	then there exists some $S \in Syl_p(G)$ with $P \leq S.$
\end{theorem}

\begin{theorem}[Sylow Conjugacy]
Let $G$ be a finite group. The set $Syl_p(G)$ is a single conjugacy class of Sylow
subgroups of $G.$
\end{theorem}

\begin{theorem}[Conjugacy of Sylow $p$\nobreakdash-subgroups]
	Let $G$ be a finite group and suppose $P \leq G$
	is a $p$\nobreakdash-subgroup and $S \in Syl_p(G).$
	Then $P \leq S^x = x^{-1}S x $ for some $x$ in $G.$
\end{theorem}

\begin{proof}
	Let $ \Omega  =  \left\{ S x \; | \; x \in G  \right\}$
	be the set of right cosets of $X$ in $G.$
	Let $P$ act on $ \Omega $ via right regular  action.
	We proceed to calculate the size of $ \Omega.$
	The size of $ \Omega $ is the sum of sizes of all orbits in it. 
	It is also equal to $ \lvert \Omega \rvert = \frac{ \lvert P \rvert} { \lvert G_S \rvert}.$
	Consequently, $p \not | \, \lvert \Omega \rvert.$

	We claim that there is an orbit of size $1.$
	Indeed, if the size of all orbits is divisible by $p,$ then $ \lvert G \rvert $
	is divisible by $p $ because the orbits of the action of $P$
	on $ \Omega$ partition $G.$
	In other words, there exists some $x \in G$ such that $ S{xy} = S {x} \cdot y = S_x$
	for each $y \in P.$
	Then $y \in x^{-1} S x = S^x.$ Thus $ P\leq S^x.$
\end{proof}

The collection of all Sylow $p$\nobreakdash-subgroups, $Syl_p(G),$ is quite interesting.

\begin{corollary}[Number of Sylow $p$\nobreakdash-subgroups] \label{cor:number-of-sylow-p-groups}
	Let $G$ be a finite group and let $P \in Syl_p(G).$ Then, $ \lvert Syl_p(G) \rvert 
	= \left[ G \, : \, N_G(P) \right] ,$
	where $N_G(P) $ is the normaliser of $P$ in $G.$
	In particular, $ \lvert Syl_p(G) \rvert $ divides $ \left[ G \, : \, P \right] 
	= \frac{ \lvert G \rvert } { \lvert P \rvert }.$
\end{corollary}

\begin{proof}
	Let $P$ act on $Syl_p(G)$ by conjugation.
	The number of conjugate subgroups of $P$ is exactly the number of 
	elements in $Syl_p(G)$
	because each Sylow $p$\nobreakdash-subgroup of $G$ is conjugate to $P.$
	Let $ \Omega$ be the set of all conjugates of $P$ in $G$ and let 
	$G$ act on $ \Omega$ by conjugation.
	The size of $ \Omega ,$ which is the number of conjugates of $P$ in $G$ 
	is $ \left[ G \, : \, N_G(H) \right] $ because $N_G(P) $ is the
	stabiliser of $P$ in $G.$
\end{proof}

\begin{corollary}[Normal Sylow $p$\nobreakdash-subgroups]
	Let $ S$ be a Sylow $p$\nobreakdash-subgroup of $G.$
	The following statements are equivalent
	\begin{enumerate}
		\item $S$ is normal in $G.$
		\item S is the unique Sylow $p$\nobreakdash-subgroup of $G.$
		\item Every $p$\nobreakdash-subgroup of $G$ is contained in $S.$
		\item $S$ is a characteristic subgroup of $G.$
	\end{enumerate}
\end{corollary}

\begin{definition}[Characteristic subgroups]
	If $G$ is a group, then a subgroup $H$ of $G$ is said to be characteristic
	in $G$ if for any $ \sigma \in \Aut(G),$ $ \sigma(S) = S.$
\end{definition}

\begin{proof}
	$ (1) \implies (2) $
	As any two Sylow $p$\nobreakdash-subgroups of $G$ are conjugate to each other,
	we conclude that there is only one Sylow $p$\nobreakdash-subgroup of $G$
	because it has only one conjugate.
	
	$(2) \implies (3) $
	As each $p$\nobreakdash-subgroup is ocntained in some Sylow $p$\nobreakdash-subgroup,
	we get that any given $p$\nobreakdash-subgroup $P$ is contained in $S.$
	
	$(3) \implies (4).$
	For each $ \sigma \in \Aut(G),$
	$ \sigma (S) $ is a $p$\nobreakdash-subgroup of $G.$
	From (3), we get that $ \sigma(S) \leq S.$
	Moreover, $ \sigma(S) = S.$ Then $S$ is characteristic in $G.$

	$(4) \implies (1) $
	This is true because conjugation is an inner automorphism.
\end{proof}

\begin{lemma}[$p$\nobreakdash-subgroups of normaliser] \label{lem:p-subgroup-of-normaliser}
	Let $G$ be a finite group and let $S \in Syl_p(G).$
	Suppose $P$ is a $p$\nobreakdash-subgroup of $N_G(S).$
	Then $P \leq S.$
\end{lemma}

\begin{proof}
	By the Sylow's Development Theorem, we know that $P \leq S^x $
	for some $x \in G.$
	Since $P \leq N_G(S),$ we know that $PS = SP.$
	From previous knowledge, we know that this happens if and only if 
	$SP$ is a subgroup of $G.$
	So, we get that $S \leq SP \leq G.$
	For any two groups $H, K $ of a group $G,$ we have
	$ \lvert H K \rvert = \frac{ \lvert H \rvert \lvert K \rvert}{ \lvert H \cap K \rvert}.$
	As $SP$ is also a $p$\nobreakdash-subgroup and
	$S$ is a maximal $p$\nobreakdash-subgroup in $G,$ we get that 
	$SP = S$ which happens if and only if $S = P.$
\end{proof}

\begin{proof}[Alternative Proof]
	Since $S \in Syl_p(G)$ and $S \leq N_G(S) \leq G,$ we must have
	$S$ is a Sylow $p$\nobreakdash-subgroup of $N_G(S).$	
	Moreover, $P$ is a $p$\nobreakdash-subgroup of $N_G(S)$
	and $S \triangleleft N_G(S)$ so applying the corollary about normal Sylow
	$p$\nobreakdash-subgroups, we get that $P \leq S.$
\end{proof}

